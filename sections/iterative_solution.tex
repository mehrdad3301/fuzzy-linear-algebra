
	\subsection{روش های تکراری}
	در جبرخطی دیدیم معکوس یک ماتریس معمولا هزینه سنگین محاسباتی دارد و برای حل این مشکل از روش های تکراری برای حل دستگاه معادلات استفاده کردیم. در این بخش قصد داریم به معرفی روش های تکراری برای حل دستگاه های فازی بپردازیم.
	 
	\subsubsection{همگرایی}
	\begin{theorem} 
	\label{thm:3}
		اگر ماتریس ضرایب $ A $ غالب قطری اکید باشد،‌ هر دو روش ژاکوبی و گاوس-سایدل در حل معادله دستگاه \ref{eq:1} همگرا هستند.\cite{second_course}
	\end{theorem}
	\begin{theorem} 
	ماتریس $ A $ غالب قطری اکید است اگر و تنها اگر ماتریس $ S $ غالب قطری اکید باشد. 
	\end{theorem}
	\textbf{اثبات:}
	با توجه به مطالب گفته شده در قسمت قبل ماتریس $ B $ و $ C $ به ترتیب عناصر مثبت و منفی $ A $ هستند و اگر درایه ای در $ C $ ناصفر باشد درایه متناظر با آن در $ B $ حتما صفر است. به طور مشابه اگر درایه ای نامنفی در $ B $ باشد درایه متناظر آن در $ C $ حتما صفر است. در واقع با قراردادن ماتریس $ B $ و $ C $ روی یکدیگر قدرمطلق ماتریس $ A $ را به دست آورده ایم. به این ترتیب واضح است که اگر $ A $ غالب قطری اکید باشد،‌$ S $ هم غالب قطری اکید است و بالعکس.  
	\subsubsection{ژاکوبی}
	بدون از دست رفتن کلیت، فرض کنید 
	$ s_{ii} > 0, 1 \leq i \leq 2n $ 
	.قرار می دهیم 
	$ S = D + L + U $ 
	که در آن  \\
	\begin{center} 
	$
	D = 
	\begin{pmatrix}
	D_B & 0 \\
	0 & D_B
	\end{pmatrix}
	\quad 
	L = 
	\begin{pmatrix}
	L_B & 0 \\
	C & L_B
	\end{pmatrix}
	\quad
	U = 
	\begin{pmatrix}
	U_B & C \\
	0 & U_B
	\end{pmatrix}
	$
	\end{center} 
	دستگاه $ SX = Y $ را به فرم بلوکی زیر می نویسیم. 
	
	\begin{center}
	$
	\begin{pmatrix}
	B & C\\ 
	C & B 
	\end{pmatrix} 
	\begin{pmatrix}
	\underline{X}\\ 
	-\overline{X}
	\end{pmatrix} 
	=
	\begin{pmatrix}
	\underline{Y}\\ 
	-\overline{Y}
	\end{pmatrix} 
	$
	
	\end{center}
	می توان فرم بالا را به صورت زیر نمایش داد. 
	
	\begin{center}
	$
	\begin{pmatrix}
	B & -C\\ 
	-C & B 
	\end{pmatrix} 
	\begin{pmatrix}
	\underline{X}\\ 
	\overline{X}
	\end{pmatrix} 
	=
	\begin{pmatrix}
	\underline{Y}\\ 
	\overline{Y}
	\end{pmatrix} 
	$
	
	\end{center}
	با توجه به فرم ژاکوبی داریم 
	
	\begin{center} 
	$
	\begin{pmatrix}
	D_B & 0 \\
	0 & D_B 
	\end{pmatrix}
	\begin{pmatrix}
	\underline{X}\\
	\overline{X} 
	\end{pmatrix}
	+ 
	\begin{pmatrix}
	L_B + U_B & -C \\ 
	-C & L_B + U_B
	\end{pmatrix}
	\begin{pmatrix}
	\underline{X}\\
	\overline{X} 
	\end{pmatrix}
	=
	\begin{pmatrix}
	\underline{Y}\\
	\overline{Y} 
	\end{pmatrix}
	$\\
	\end{center} 
	با ضرب در $ D_B^{-1} $ روابط تکراری زیر را به دست می آوریم
	\begin{center}
	$\underline{X}^{k+1} = D_B^{-1}\underline{Y} - D_B^{-1}(L_B + U_B)\underline{X}^k + D_B^{-1}C\overline{X}^k$	
	$\overline{X}^{k+1} = D_B^{-1}\overline{Y} - D_B^{-1}(L_B + U_B)\overline{X}^k + D_B^{-1}C\underline{X}^k$
	\end{center} 

	می توان روابط بالا را به فرم فشرده ماتریسی هم نوشت. بدین صورت که 
	$
	X^{k+1} = M_J X^{k} + C_J 
	$
	که $ M_J $ و $ C_J $ به فرم زیر هستند 
	
	
	\begin{center}
		$
		M_J =
		\begin{pmatrix} 
		-D_B^{-1}(L_B + U_B) & D_B^{-1}C \\
		D_B^{-1}C & -D_B^{-1}(L_B + U_B) 
		\end{pmatrix} 
		\quad 
		C_J = 
		\begin{pmatrix}
		D_B^{-1}\underline{Y} \\
		D_B^{-1}\overline{Y} 
		\end{pmatrix} 
		\quad
		X = 
		\begin{pmatrix}
		\underline{X}\\ 
		\overline{X}
		\end{pmatrix}
		$
	\end{center}

	\pagebreak  

	\subsubsection{گاوس-سایدل}
	با رویکردی مشابه روش ژاکوبی روابط تکراری را برای روش گاوس-سایدل به دست می آوریم. ابتدا مشابه $ SX = Y $ را برای فرم گاوس سایدل بازنویسی می کنیم. 
	 
	 \begin{center} 
	 	$
	 	\begin{pmatrix}
	 	L_B + D_B & 0 \\
	 	-C & L_B + D_B 
	 	\end{pmatrix}
	 	\begin{pmatrix}
	 	\underline{X}\\
	 	\overline{X} 
	 	\end{pmatrix}
	 	+ 
	 	\begin{pmatrix}
	 	U_B & -C \\ 
	 	0 & U_B
	 	\end{pmatrix}
	 	\begin{pmatrix}
	 	\underline{X}\\
	 	\overline{X} 
	 	\end{pmatrix}
	 	=
	 	\begin{pmatrix}
	 	\underline{Y}\\
	 	\overline{Y} 
	 	\end{pmatrix}
	 	$\\
	 \end{center}
 	کافیست معادلات بالا را به صورت تکراری بازنویسی کنیم 
 	\begin{center}
 		$\underline{X}^{k+1} = \left(L_B + D_B\right)^{-1}\underline{Y} - \left(L_B + D_B\right)^{-1}U_B\underline{X}^k + \left(L_B + D_B\right)^{-1}C\overline{X}^k$	
 		 $\overline{X}^{k+1} = \left(L_B + D_B\right)^{-1}\overline{Y} - \left(L_B + D_B\right)^{-1}U_B\overline{X}^k + \left(L_B + D_B\right)^{-1}C\underline{X}^k$	
 	\end{center} 
 	مشابه قسمت قبل می توان روابط بالا را به فرم ماتریسی هم بازنویسی کرد. به طوری که \\
 	$ X^{k+1} = M_{GS}X^k + C_{GS} $ 
 	که در آن $M_{GS} $ و $ C_{GS}$ به صورت زیر تعریف می شوند \\
 	
 	$ 
 	M_{GS} = 
 	\begin{pmatrix}
 	- \left(L_B + D_B\right)^{-1}U_B & \left(L_B + D_B\right)^{-1}C \\ 
    \left(L_B + D_B\right)^{-1}C & - \left(L_B + D_B\right)^{-1}U_B
 	\end{pmatrix}
 	\quad
 	C_{GS} = 
 	\begin{pmatrix}
 	\left(L_B + D_B\right)^{-1}\underline{Y}\\
 	\left(L_B + D_B\right)^{-1}\overline{Y}
 	\end{pmatrix} \\ 	
 	$	
 	
 	اگر شرایط قضیه \ref{thm:3} برقرار باشد، هر دو روش برای هر $ X^0 $ دلخواهی همگرا هستند.
 	
 	\subsubsection{معیار توقف}
 	در سیستم های فازی با توجه به اینکه اعداد فازی به صورت زوج های مرتب هستند،‌ طبیعی است که معیار توقف به صورت دو نامساوی باشد و زمانی توقف کنیم که شرایط هر دو نامساوی برقرار شده باشد. 
 	\begin{align}
 	\frac{||\overline{X}^{k+1} - \overline{X}^k||}{||\overline{X}^k||} < \epsilon 
 	\quad 
 	\frac{||\underline{X}^{k+1} - \underline{X}^k||}{||\underline{X}^k||} < \epsilon 
 	\end{align}
 	
	