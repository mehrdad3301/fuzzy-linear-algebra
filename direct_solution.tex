	\section{حل دستگاه های فازی}
 	در این قسمت ابتدا طبق رویکرد \cite{friedman} روشی عمومی برای حل مستقیم دستگاه ارائه می کنیم. در بخش ۲ به بررسی روش های تکراری گاوس-سایدل و ژاکوبی می پردازیم. در بخش آخر به ارائه مثال های عددی می پردازیم. 
 	
	\subsection{روش مستقیم}
	در اینجا با تبدیل دستگاه $ n \times n $ به دستگاه قطعی $ 2n \times 2n $ روشی برای حل دستگاه فازی ارائه می کنیم. 
	در واقع ما قصد داریم دستگاه را به صورتی بازنویسی کنیم که دیگر عنصر فازی نداشته باشیم. به عبارتی عناصر دستگاه جدید دیگر به فرم زوج مرتبی نخواهند بود و به راحتی می توانیم دستگاه جدید را حل کنیم. 
	بردار 
	$ y = (\underline{y}_1, \underline{y}_2, \ldots, \underline{y}_n, \overline{y}_1, \ldots, \overline{y}_n)^T $
	را بردار سمت راست و 
	$ \underline{x}_i, \left(-\overline{x}_i\right)$
	متغیر های دستگاه قرار می دهیم. دستگاه معادلات جدید را به صورت زیر فرمول بندی می کنیم. 
	\begin{flalign}
	\begin{split}
		\underline{y}_1 & = s_{11}\underline{x}_1 + \ldots + s_{1n}\underline{x}_n +
		s_{1,n+1}\left(\overline{x}_1\right) + \ldots + s_{1,2n}\left(\overline{x}_n\right)\\
		\underline{y}_2 & = s_{21}\underline{x}_1 + \ldots + s_{2n}\underline{x}_n +
		s_{2,n+1}\left(\overline{x}_1\right) + \ldots + s_{2,2n}\left(\overline{x}_n\right)\\
		\vdots\\ 
		\underline{y}_n & = s_{n1}\underline{x}_1 + \ldots + s_{nn}\underline{x}_n + 
		s_{n,n+1}\left(\overline{x}_1\right) + \ldots + s_{n,2n}\left(\overline{x}_n\right)\\
		-\overline{y}_1 & = s_{n+1,1}\underline{x}_1 + \ldots + s_{n+1,n}\underline{x}_n +
		s_{n+1,n+1}\left(\overline{x}_1\right) + \ldots + s_{n+1,2n}\left(\overline{x}_n\right)\\
		-\overline{y}_2 & = s_{n+2,1}\underline{x}_1 + \ldots + s_{n+2,n}\underline{x}_n
		s_{n+2,n+1}\left(\overline{x}_1\right) + \ldots + s_{n+2,2n}\left(\overline{x}_n\right)\\
		\vdots\\ 
		-\overline{y}_n & = s_{2n,1}\underline{x}_1 + \ldots + s_{2n,n}\underline{x}_n + 
		s_{2n,n+1}\left(\overline{x}_1\right) + \ldots + s_{2n,2n}\left(\overline{x}_n\right)\\
	\end{split}
	\end{flalign} 
	که $ S $ به صورت زیر به دست می آیند 
	\begin{equation}
	\label{eq:4}
		\begin{cases}
		\text{$s_{ij} = a_{ij}, \quad s_{i+n,j+n} = a_{ij}$} &\quad\text{$a_{ij} \geq 0$}\\
		\text{$s_{i,j+n} = -a_{i,j}, \quad s_{i+n,j} = -a_{ij} $} &\quad\text{$a_{ij} < 0$}\\
		\end{cases}
	\end{equation} 
	
	توجه کنید که مقدار $ s_{ij} $ برای مقادیری که توسط رابطه بالا مشخص نمی شوند، ‌صفر است.\\
	با توجه به روابط بالا دستگاه معادلات جدید به صورت $ \mathbf{SX=Y} $ است که در آن
	$ \mathbf{S}=(s_{ij})$ \\
	$ 1 \leq i,j \leq 2n $ 
	است و \\
	\begin{center}
	$
	X = 
	\begin{pmatrix} 
	\underline{x}_1\\
	\underline{x}_2\\
	\vdots\\
	\underline{x}_n\\
	-\overline{x}_1\\
	\vdots\\
	-\overline{x}_n	
	\end{pmatrix}
	,
	Y =
	\begin{pmatrix} 
	\underline{y}_1\\
	\underline{y}_2\\
	\vdots\\
	\underline{y}_n\\
	-\overline{y}_1\\
	\vdots\\
	-\overline{y}_n	
	\end{pmatrix}
	$
	\end{center}

	\theoremstyle{definition}
	\newtheorem{exm}{Theorem}[section]
	\newtheorem{example}[exm]{مثال}
	\begin{example}
		دستگاه  $2 \times 2 $ زیر را در نظر بگیرید. \\
	\begin{flalign}
		x_1 - x_2  & = y_1 \nonumber\\
		x_1 + 2x_2 & = y_2 \nonumber
	\end{flalign}
	\end{example}
	با توجه به روابط \ref{eq:4} می توان ماتریس S را به صورت زیر ساخت.\\
	\begin{center}
		$
		\begin{pmatrix}
		1 & 0 & 0 & 1 \\
		1 & 2 & 0 & 0 \\
		0 & 1 & 1 & 0 \\
		0 & 0 & 1 & 2 
		\end{pmatrix}
		$
	\end{center}

	به راحتی می توان دید که فرم ماتریس S به صورت زیر است.\\
	\begin{center}
		$
		S = 
		\begin{pmatrix}
		B & C \\
		C & B \\
		\end{pmatrix}
		$
	\end{center}
	که در آن $ B $ شامل درایه های مثبت ماتریس ضرایب $ A $ و $ C $ از قدرمطلق درایه های منفی تشکیل شده. پس می توان گفت $ A = B - C $.\\
	دستگاه جدیدی که به دست آوردیم تماما از درایه های قطعی تشکیل شده. تنها در صورتی می توان برای این دستگاه جواب یکتا به دست آورد که ماتریس $ S $ نامنفرد باشد. در اینجا با دو سوال رو به رو می شویم. 
	\begin{enumerate}
		\item آیا ماتریس $ S $ نامنفرد است؟
		\item آیا جواب به دست آمده می تواند تشکیل برداری فازی برای دستگاه اولیه دهد؟‌
	\end{enumerate}

	اگر که پاسخ سوال ۱ مثبت باشد،‌ پاسخ سوال دوم در صورتی مثبت است که برای هر $ i $,
	$ \left(\underline{x}ـ{i}\left(r\right), \overline{x}_{i}\left(r\right)\right) $
	عددی فازی باشد. 
	
	در ادامه ابتدا به سوال اول پاسخ می دهیم. باید بررسی کنیم که تحت چه شرایطی ماتریس $ S $ نامنفرد است. از آنجایی که $ S $ را با استفاده از ماتریس ضرایب $ A $ ساختیم، حدس می زنیم که روابطی بین معکوس پذیری $ A $ و $ S $ وجود داشته باشد. مثال بعد نشان می دهد که نامنفرد بودن $ A $ شرط کافی برای معکوس پذیری $ S $ نیست. \\
	\begin{example}

		دستگاه  $2 \times 2 $ زیر را در نظر بگیرید. \\
		\begin{flalign}
		x_1 - x_2  & = y_1 \nonumber\\
		x_1 + x_2 & = y_2 \nonumber
		\end{flalign}
	
	\end{example}
	مشابه مقال قبل ماتریس $ S $ را به دست می آوریم. 

	\begin{center}
	$
	S = 
	\begin{pmatrix}
	1 & 0 & 0 & 1 \\
	1 & 1 & 0 & 0 \\
	0 & 1 & 1 & 0 \\
	0 & 0 & 1 & 1 
	\end{pmatrix}
	$
	\end{center}
	
	می توان دید که $ S $ وارون پذیر نیست. 
	در قضیه بعد شرط لازم و کافی برای معکوس پذیری $ S $ را بیان میکنیم. 
	\newtheorem{thm}{Theorem}[section]
	\newtheorem{theorem}[thm]{قضیه}
	\begin{theorem}
	\label{thm:1}
			ماتریس $ S $ نامنفرد است اگر و تنها اگر ماتریس $ A = B - C $ و $ B + C $ نامنفرد باشند. \\
	\textbf{اثبات:}
			با توجه به اینکه با جمع ترکیب خطی از سطر ها یا ستون های یک ماتریس تاثیری در دترمینان آن ندارد، می توان به راحتی روابط زیر را به دست آورد.

	\begin{center}	
	$ 
	S = 
	\begin{pmatrix}
	B & C \\
	C & B \\
	\end{pmatrix} 
	\rightarrow 
	S_1 = 
	\begin{pmatrix}
	B + C & B + C \\
	C & B \\
	\end{pmatrix} 
	\rightarrow 
	S_2 = 
	\begin{pmatrix}
	B + C & 0 \\
	C & B - C \\
	\end{pmatrix} 
	$\\ [10pt]
	
	$det(S) = det(S_1) = det(S_2) = det(B + C).det(B - C) = det(B + C).det(A)$\\[10pt]
	$\rightarrow \quad\quad det(S) \neq 0 \longleftrightarrow det(A) \neq 0 \quad and \quad det(B + C) \neq 0$	
	\end{center}
	\end{theorem}
	قضیه \ref{thm:1} از اهمیت فراوانی برخوردار است از آنجایی که شرطی لازم و کافی برای معکوس پذیری $ S $ بیان می کند. برای حل دستگاه فازی بعد از اینکه فهمیدیم $ S $ معکوس پذیر است باید $ S^{-1} $ را به دست بیاوریم. در ادامه به بیان قضایای و روش هایی برای به دست آوردن $ S^{-1} $  می پردازیم. 
	
	\begin{theorem}
	\label{thm:2} 
		وارون ماتریس $ S $ در صورت وجود، فرمی مشابه $ S $ دارد. 
	\[
		S^{-1} = 
		\begin{pmatrix}
		E & D\\ 
		D & E
		\end{pmatrix}		
	\]
	
	\end{theorem}
	برای دیدن اثبات قضیه فوق به \cite{friedman} مراجعه کنید. 
	با توجه به قضیه \ref{thm:2} $ S^{-1} $ را به صورت زیر به دست می آوریم. 
	
	\[
	SS^{-1} = 
	\begin{pmatrix}
	B & C \\ 
	C & B \\  
	\end{pmatrix}
	\begin{pmatrix}
	D & E \\ 
	E & D \\
	\end{pmatrix}
	\]
	\begin{align}
	\label{eq:5}
	BD + CE = I, \quad CD + BE = 0 
	\end{align} 
	با جمع و تفریق دو معادله در رابطه \ref{eq:5} روابط زیر را به دست می آوریم. 
	\begin{align}
	\label{eq:6}
	D + E = (B + C)^{-1}, \quad D - E = (B - C)^{-1}
	\end{align} 
	و نتیجتا 
	\begin{align}
	\label{eq:7}
	E = \frac{1}{2}\lbrack(B + C)^{-1} - (B - C)^{-1}\rbrack, \quad 
	D = \frac{1}{2}\lbrack(B + C)^{-1} + (B - C)^{-1}\rbrack
	\end{align} 
	با توجه به فرض نامنفرد بودن $ S $، که معادل با نامنفرد بودن $ B + C $ و $ B - C $است،‌ داریم 
	\begin{align}
	\label{eq:8}
	X = S^{-1}Y
	\end{align} 
	که ممکن است $ X $ برداری فازی نباشد. قضیه بعد شرط لازم و کافی برای فازی بودن $ X $ را بیان می کند. 
	
	\begin{theorem} 
		بردار $ X $ ر رابطه \ref{eq:8} فازی است، ‌اگر و تنها اگر $ S^{-1} $ مثبت باشد، به عبارتی باید داشته باشیم
	\begin{align}
		\left(S^{-1}\right)_{ij} \geq 0, \quad 1 \leq i,j \leq n 
	\end{align}
	\end{theorem}
	\textbf{اثبات:} 
	قرار می دهیم 
	$ .S^{-1} = \left(t\right)_{ij}, 1 \leq i,j \leq n$ 
از رابطه \ref{eq:8} داریم 
 	\begin{align}
		\underline{x}_i & = \sum_{j=1}^n t_{ij}\underline{y}_i - \sum_{j=1}^n t_{i,n+j}\overline{y}_i, \quad 1 \leq i \leq n\\
		-\overline{x}_i & = \sum_{j=1}^n t_{n+i,j}\underline{y}_i - \sum_{j=1}^n t_{n+i,n+j}\overline{y}_i, \quad 1 \leq i \leq n
	\end{align}
	با توجه به قضیه \ref{thm:2} می توان معادلی برای رابطه ۱۱ به صورت زیر نوشت 
	\begin{align}
	\overline{x}_i & = -\sum_{j=1}^n t_{i,n+j}\underline{y}_i + \sum_{j=1}^n t_{i,j}\overline{y}_i, \quad 1 \leq i \leq n
	\end{align}
	با کم کردن رابطه۱۰ از ۱۲ رابطه زیر را به دست می آوریم 
	\begin{align}
	\overline{x}_i - \underline{x}_i & = \sum_{j=1}^n t_{i,j}\left(\overline{y}_i - \underline{y}_i\right) + \sum_{j=1}^n t_{i,n+j}\left(\overline{y}_i - \underline{y}_i\right), \quad 1 \leq i \leq n
	\end{align}
	از آنجایی که $ y $ برداری فازی است،‌ برای هر $ 0 \leq i \leq n $ داریم 
	$ .\overline{y}_i - \underline{y}_i \geq 0$
	در نتیجه با توجه به رابطه ۱۳ می توان گفت 
	$ \overline{x}_i - \underline{x}_i \geq 0$ 
	اگر و تنها اگر برای هر $ 0 \leq j \leq n $ داشته باشیم
	$ .t_{ij} \geq 0, 1 \leq j \leq n $
	پس ثابت کردیم که شرط سوم فازی بودن برقرار است. بعلاوه از روابط ۱۰ و ۱۲ و با توجه به اینکه $ \overline{y}_i $ نزولی یکنوا و 
	$ \underline{y}_i $ صعودی یکنواست 
	می توان گفت شرط ۱ و ۲ هم برای $ \underline{x}_i $ و $ \overline{x}_i $ برقرار است. دقت کنید که از چپ و راست پیوسته و کراندار بودن $ \underline{x}_i $ و $ \overline{x}_i $ به وضوح برقرار است چرا که، با توجه به روابط ۱۰ و ۱۲، ترکیبی خطی از $ \underline{y}_i $ و $ \overline{y}_i $ ها هستند.\\~\\
	متاسفانه احتمال اینکه $ S^{-1} $ شروط قضیه قبل را دارا باشد در عمل کم است. \\
	\subsection{یک روش جایگزین}
	از آنجایی که تشکیل یک ماتریس $ 2n \times 2n $ ممکن از لحاظ محاسباتی هزینه بر باشد،‌ عزتی \cite{ezzati} با اقتباس از این روش،‌ روشی جدید برای حل دستگاه های فازی ارائه داد که بجای حل یک سیستم $ 2n \times 2n $ با حل دو دستگاه $ n \times n $ جواب دستگاه معادلات را به دست می آورد. \\

	